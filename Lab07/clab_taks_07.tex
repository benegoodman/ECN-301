% -------------------------------------------
% -- ECN 301 HW Set 0                      --
% -------------------------------------------

\documentclass[a4paper,10pt]{lec_nmbu}
\usepackage[utf8]{inputenc}
\usepackage[american]{babel}
\usepackage{amsmath}
\usepackage{fancyhdr}
\usepackage{natbib}
\usepackage{url}
\usepackage{booktabs}
\usepackage{graphicx}
\usepackage{pgfpages}

\normalfont
\usepackage[T1]{fontenc}
\usepackage{textcomp}

\usepackage{courier}
\usepackage[scaled]{helvet}
\usepackage[sc,osf]{mathpazo}
\usepackage[charter]{mathdesign}

\linespread{1.05}


% -- definer overskrifter
\leccode{ECN 301}
\title{Econometric Methods}
\lectype{Computer Lab Task \#07}
\lecdate{Spring Term 2022}
\author{Olvar Bergland}

% define Fancyheaders
\pagestyle{fancy}
\addtolength{\headwidth}{\marginparsep}
\addtolength{\headwidth}{\marginparwidth}
\addtolength{\headheight}{4pt}
\lhead{\bfseries{ECN 301}}
\chead{\bfseries{Computer Lab \#07}}
\rhead{\bfseries\thepage}
\lfoot{}
\cfoot{}
\rfoot{}


\begin{document}

\maketitle


The study by \citet{ODO06EST} uses a dataset consisting of information collected
from 44 smallholder rice producers in the Tarlac region of the Philippines
between 1990 and 1997\footnote{This is panel data. However, here we will treat it as a
random sample. I will revisit this dataset later in the course.}. The dataset is
also used in the textbooks by \citet{COE05INT} and \citet{HIL18PRI} to
illustrate estimation of production and cost functions. The dataset, called
\texttt{rice2}, includes these variables:

\begin{footnotesize}
\begin{verbatim}
-------------------------------------------------------
Contains data from rice2.csv
-------------------------------------------------------
Philippines Rice Data, IRRI
10 Oct 2013 11:04
-------------------------------------------------------
  obs:           352
 vars:             6
-------------------------------------------------------
variable name   variable label
-------------------------------------------------------
farmid          Farm ID
year            Year
prod            Rice production (tonnes)
area            Area planted to rice (hectares)
labor           Labor (man-days of hired + family)
fert            Fertilizer applied (kilograms)
-------------------------------------------------------
Sorted by:  farmid  year
-------------------------------------------------------
\end{verbatim}
\end{footnotesize}

Consider the following Cobb-Douglas production function specification:
\begin{equation}
   \ln q_{it} = \alpha + \sum_{j=1}^{3} \beta_j \ln x_{jit} + \varepsilon_{it}
   \quad i=1,\ldots,N \ t?=1.\ldots,T.
   \label{fn:cd}
\end{equation}
where $\alpha$ represents the constant term and$\varepsilon_{it}$ is the idiosyncratic error term.
The output level (\texttt{prod}) is $q_{it}$, and the level of the factors of production are denoted
$x_{jit}$ for factor $j$ for farmer $i$ in year $t$. The three factors of
production are land (\texttt{area}), \texttt{labor} and fertilizer (\texttt{fert}).
The number of observations is $NT$.

\begin{enumerate}
   \item Verify the number entities (farmers) and the number of time periods.

   \item Create dummy variables for the years. Perform the estimation of the POLS (see below) both
         using explicit dummy variables and with year as a category variable.

   \item The key task here is to use a variety of estimators for this data:
         \begin{enumerate}
          \item POLS
          \item FE
          \item RE
          \item CRE
         \end{enumerate}

         And then provide a good argued answer to:
         \begin{enumerate}
          \item are there unobserved individual effects
          \item are there year specific effects
          \item which estimator is valid/preferred and why
         \end{enumerate}

   \item Finally, is there constant returns-to-scale in rice farming?

\end{enumerate}

\bibliographystyle{econometrica}
\bibliography{refbase}

\end{document}
\endinput

