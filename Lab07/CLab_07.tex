% --------------------------------------------------------------------
% -- ECN 301 Computer Lab #01                                       --
% --     Olvar Bergland, 2014 NMBU Total revision                   --
% --------------------------------------------------------------------

\documentclass[aspectratio=169,t]{beamer}
%\documentclass[draft]{beamer}
%\documentclass[handout,t]{beamer}

\usepackage[utf8]{inputenc}
\usepackage{amsmath,amssymb}
\usepackage{url}
\usepackage{graphicx}
\usepackage{booktabs}
\usepackage{pgfpages}

\usepackage{bm}
\usepackage{eulervm}
\usepackage[default]{lato}

\usepackage{hyperref}
\hypersetup{%
   colorlinks  = {true},
   urlcolor    = {blue},
   linkcolor   = {black},
   citecolor   = {black},
   pdfauthor   = {Olvar Bergland},
   pdftitle    = {ECN301: Econometric Methods},
   pdfkeywords = {BibTeX, LaTeX}
}

\usetheme{Boadilla}
\usecolortheme[RGB={0,154,129}]{structure} % NMBU official color code
\usefonttheme{professionalfonts}
\setbeamerfont{note page}{family*=pplx}
\setbeamertemplate{note page}[plain]
\addtobeamertemplate{note page}{\setlength{\parskip}{12pt}}

%
% options for presentation mode only
\mode<beamer>{
  \setbeamercovered{transparent}
  \hypersetup{pdfpagemode={FullScreen},%
              bookmarksopen,%
              bookmarksopenlevel=1}
  \setbeameroption{hide notes}
}

%
% options for printed handouts (with my notes!)
\mode<handout>{
  \usepackage{handoutWithNotes}
  \pgfpagesuselayout{4 on 1 with notes}[a4paper,border shrink=5mm]
  \pgfpageslogicalpageoptions{1}{border code=\pgfusepath{stroke}}
  \pgfpageslogicalpageoptions{2}{border code=\pgfusepath{stroke}}
  \pgfpageslogicalpageoptions{3}{border code=\pgfusepath{stroke}}
  \pgfpageslogicalpageoptions{4}{border code=\pgfusepath{stroke}}
  \setbeameroption{show notes}
}


\pgfdeclareimage[height=10mm]{nmbu-logo}{../../picts/NMBU_symbol_RGB}
\logo{\pgfuseimage{nmbu-logo}}


% ----------------------------------------------------------
% --  Lecture identifications                             --
% ----------------------------------------------------------

\title[ECN 301: Computer Lab 07] %
      {ECN 301: Computer Lab \# 07\\
       Introduction to Python and Scripting}

\author{Olvar Bergland}


\institute[NMBU] %
  {
  School of Economics and Business\\
  Norwegian University of Life Sciences (NMBU) \\
  and \\
  School of Economic Sciences\\
  Washington State University (WSU)
  }


\date{January 2022}


\begin{document}

\begin{frame}
 \titlepage
\end{frame}

%\section*{Outline}

\begin{frame}
  \frametitle{Preparation}

  Before this computer lab:
  \begin{itemize}
   \item Install Python on your computer
   \item Look at the first five sections in ``Learning Python the Hard Way'' (oops, no longer freely available)
   \item Look at the first chapter in ``Using Python for Introductory Econometrics'' (\url{http://www.upfie.net/})
   \item Look at automate the boring stuff \url{https://automatetheboringstuff.com/}
  \end{itemize}

\end{frame}

\begin{frame}
  \frametitle{Python Distributions}

  Python is freely available, and provided by (among others):
  \begin{itemize}
   \item anaconda
   \item Microsoft
   \item Linux (depend on distribution)
   \item Apple
  \end{itemize}

  Many prefer anaconda irrespective of operating system

  (\url{https://www.anaconda.com/products/individual})

\end{frame}


\begin{frame}
  \frametitle{Python Tools}

  Python development tools:
  \begin{itemize}
   \item IDE (integrated development environment)
     \begin{itemize}
      \item Visual Studio Code (Microsoft)
      \item PyCharm (\url{https://www.jetbrains.com/pycharm/})
      \item Spyder (\url{https://www.spyder-ide.org/})
      \item Jupyter Notebook (not really IDE)
     \end{itemize}
   \item editor
     \begin{itemize}
      \item Notepad++
      \item Atom (\url{https://atom.io/})
     \end{itemize}
  \end{itemize}

  Many prefer anaconda irrespective of operating system

  (\url{https://www.anaconda.com/products/individual})

\end{frame}


\begin{frame}
  \frametitle{Python Environment}

  Python lives in part on the command line:
  \begin{itemize}
   \item Anaconda includes a power shell
   \item Windows also has a power shell
   \item this is old news in Linux
   \item create at least one environment
     \begin{itemize}
      \item insulates and specializes Python
      \item (\url{https://hackmd.io/u06NnoS9RrCE6Yp8aj9BpQ})
     \end{itemize}
  \end{itemize}


F. Vella and M. Verbeek (1998), "Whose Wages Do Unions Raise? A Dynamic Model
of Unionism and Wage Rate Determination for Young Men," Journal of Applied
Econometrics 13, 163-183.

nr                       person identifier
year                     1980 to 1987
black                    =1 if black
exper                    labor market experience
hisp                     =1 if Hispanic
hours                    annual hours worked
married                  =1 if married
educ                     years of schooling
union                    =1 if in union
lwage                    log(wage)
expersq                  exper^2
occupation               Occupation code


\end{frame}



\begin{frame}
  \frametitle{Python Requirements}

  The tools we will be using: Python 3.6+ and the science stack:
  \begin{enumerate}
   \item numpy
   \item pandas
   \item matplotlib
   \item scipy
   \item statsmodels
  \end{enumerate}
  as well as \texttt{linearmodels}.

\end{frame}


\begin{frame}
  \frametitle{Python Tools}

  Why the different Python packages:
  \begin{enumerate}
   \item data management: pandas
   \item creating graphs: matplotlib (and SeaBorn)
   \item statistical tools: scipy and statsmodels
   \item computation: numpy
   \item econometric models: statsmodels and linearmodels
   \item machine learning: scikit-learn
  \end{enumerate}

\end{frame}



\begin{frame}
  \frametitle{Writing Scripts}

   \begin{itemize}
    \item We will start with writing \emph{sripts} -- a list of commands
    \item Use a command window and a text editor
    \item Don't use a fancy interface
    \item Exceptions (sometimes):
      \begin{itemize}
       \item IPython
       \item Jupyter notebooks
      \end{itemize}
   \end{itemize}

\end{frame}


\begin{frame}[fragile]
  \frametitle{Silly script \# 1}

  \begin{itemize}
   \item The famous ``hello world'' program
     \begin{verbatim}
        print('Hello world')
     \end{verbatim}
   \item Save this in a text file called \texttt{hello.py}.
   \item At command prompt issue \texttt{python hello.py}:
     \begin{verbatim}
(vp3) [olvar@thinkpad code]$ python hello.py
Hello world
(vp3) [olvar@thinkpad code]$
     \end{verbatim}
  \end{itemize}

\end{frame}


\begin{frame}[fragile]
  \frametitle{Silly script \#2}

  Another silly example
  \begin{verbatim}
import datetime
print('Today is: {}'.format(datetime.datetime.today())
  \end{verbatim}

  At command prompt issue \texttt{python today.py}:
     \begin{verbatim}
(vp3) [olvar@thinkpad code]$ python hello.py
Today is: 2021-02-30 14:25:08
(vp3) [olvar@thinkpad code]$
     \end{verbatim}

\end{frame}

\begin{frame}[fragile]
  \frametitle{Reading datafiles (CSV-files)}

  \begin{enumerate}
   \item Reading datafiles are a pain, but ...
   \item Pandas has a nice smart automagical interface to csv-files
   \item Reads numbers, strings and dates
   \item Only write Python code for reading raw files if everything else fails!
  \end{enumerate}

\end{frame}

\begin{frame}[fragile]
 \frametitle{Structure of CSV-files}

 A typical CSV file may look like this:
 \begin{verbatim}
hid,total,food
1,835.94,561.08
2,876.04,617.24
3,951.47,591.24
 \end{verbatim}

 This fits with default Pandas style:
\begin{verbatim}
import pandas as pd
engel = pd.read_csv("./belgium.csv")
\end{verbatim}


\end{frame}


\begin{frame}[fragile]
 \frametitle{Structure of CSV-files}

 This fits with default Pandas style:
\begin{verbatim}
engel.head()
\end{verbatim}

This tells us:
\begin{verbatim}
 <class 'pandas.core.frame.DataFrame'>
RangeIndex: 198 entries, 0 to 197
Data columns (total 3 columns):
 #   Column  Non-Null Count  Dtype
---  ------  --------------  -----
 0   hid     198 non-null    int64
 1   total   198 non-null    float64
 2   food    198 non-null    float64
dtypes: float64(2), int64(1)
memory usage: 4.8 KB
\end{verbatim}


\end{frame}


\begin{frame}[fragile]
 \frametitle{Structure of CSV-files}

 The variable \texttt{hid} is a unique identifier - index
\begin{verbatim}
engel = engel.set_index(['hid'])
engel.head()
\end{verbatim}

This tells us:
\begin{verbatim}
<class 'pandas.core.frame.DataFrame'>
Int64Index: 198 entries, 1 to 198
Data columns (total 2 columns):
 #   Column  Non-Null Count  Dtype
---  ------  --------------  -----
 0   total   198 non-null    float64
 1   food    198 non-null    float64
dtypes: float64(2)
memory usage: 4.6 KB
\end{verbatim}

\end{frame}


\end{document}
\endinput

